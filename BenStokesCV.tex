\documentstyle[xcolor,hyperref,margin,line,pifont,times]{CV}
\special{papersize=8.5in,11in}
\definecolor{dark-blue}{rgb}{0.15,0.15,0.45}
\hypersetup{colorlinks=true,urlcolor={dark-blue}}
\setlength{\topmargin}{-.4in}
\oddsidemargin -.4in
\evensidemargin -.4in
\textwidth=6in
\textheight=9.8in
\itemsep=0in
\parsep=0in

\newenvironment{list1}{
  \begin{list}{\ding{113}}{
      \setlength{\itemsep}{0in}
      \setlength{\parsep}{0in} \setlength{\parskip}{0in}
      \setlength{\topsep}{0in} \setlength{\partopsep}{0in} 
      \setlength{\leftmargin}{0.17in}}}{\end{list}}
\newenvironment{list2}{
  \begin{list}{$\bullet$}{
      \setlength{\itemsep}{0in}
      \setlength{\parsep}{0in} \setlength{\parskip}{0in}
      \setlength{\topsep}{0in} \setlength{\partopsep}{0in} 
      \setlength{\leftmargin}{0.2in}}}{\end{list}}

\begin{document}

\name{Benjamin Stokes, Ph.D. \vspace*{.1in}}

\begin{resume}
\section{\sc Contact Information}
\vspace{.05in}
\begin{tabular}{@{}p{2.5in}p{2.5in}}
{\it Mobile:}  +1 801 839 8993 & \href{http://www.linkedin.com/in/benjamintstokes}{www.linkedin.com/in/benjamintstokes}\\
{\it E-mail:}  \href{mailto:stokes@cosmic.utah.edu}{stokes@cosmic.utah.edu} & \href{https://github.com/benjamin-stokes}{github.com/benjamin-stokes}\\  

\end{tabular}

\section{\sc Nationality}
United States Citizen 

\section{\sc Summary}
I am a problem solver with extensive experience collecting, processing, and analyzing large amounts of data in distributed settings.  Through a high degree of creative persistence, I have successfully tackled extremely difficult problems.

\section{\sc Education}
{\bf University of Utah}, Salt Lake City, Utah, USA\\
{\em Department of Physics and High Energy Astrophysics Institute} 
\begin{list1}
\item[] Ph.D., Physics, August, 2006
\begin{list2}
\item \href{http://www.cosmic-ray.org/thesis/stokes.html}{Dissertation Topic}: {\bf A Search for Anisotropy in the Arrival Directions of Ultra High Energy Cosmic Rays Observed by the High Resolution Fly's Eye Detector}
\item Advisor:  Charles C.H.\ Jui
\end{list2}
\item[] B.A., Physics, December, 1997
\end{list1}
\vspace*{.05in}

\section{\sc Honors and Awards} 
Outstanding Postdoctoral Researcher, University of Utah, Physics \& Astronomy, 2011\\
U.S.\ Presidential Scholar, Department of Education, 1994\\
National Science Scholar, United States, Department of Education, 1994\\

\section{\sc Research Experience}

{\bf University of Utah}, Salt Lake City, Utah, USA

\vspace{-.3cm}

{\em Postdoctoral Associate} \hfill {\bf May, 2010 - Present}\\
I am a research collaborator with the
    \href{http://www.telescopearray.org}{Telescope Array} (TA)
    cosmic ray observatory. As a one-of-a-kind facility, the 150
    scientists and technologists in the TA collaboration are
    responsible for design and maintenance of remote facilities (in
    Millard County, Utah), data acquisition, data storage, data
    analysis and public dissemination of resulting scientific
    knowledge.  This mission has required the TA collaboration
    to develop extensive computational resources and has resulted in
    groundbreaking discoveries about the origin and composition of
    cosmic rays.
    
{\bf Rutgers, The State University of New Jersey}, Piscataway, New Jersey, USA

\vspace{-.3cm}

{\em Postdoctoral Associate} \hfill {\bf July, 2008 - May, 2010}\\
I first joined the TA collaboration as a researcher
    with the affiliated group at Rutgers University.  While my
    position was officially in New Jersey, I spent most of my time on
    remote deployment in Utah.
    
{\bf University of Hawai`i at M\={a}noa}, Honolulu, Hawai`i, USA

\vspace{-.3cm}
{\em Junior Researcher} \hfill {\bf September, 2006 - June, 2008}\\
After my astrophysics research fellowship ended, I decided
    to remain in Honolulu and secured a position with the
    \href{http://hawaii.edu/mri/home_v6.htm}{Neuroscience and
      MRI Research Program}.  My primary responsibilities
    included analyzing fMRI data, administering a 25-node Linux
    cluster, and performing clinical duties.

{\em Research Fellow} \hfill {\bf January, 2005 - August, 2006}\\
I spent 18 months in a research fellowship studying radio
    detection of cosmic rays and neutrinos as a member of both the
    \href{http://arxiv.org/abs/astro-ph/0512265}{ANITA} and
    \href{http://arxiv.org/abs/0705.2589}{AMBER} collaborations.


{\bf University of Utah}, Salt Lake City, Utah, USA

\vspace{-.3cm}
{\em Research Assistant} \hfill {\bf June 1998 - January 2005}\\
Worked as a graduate research assistant for the High Resolution Fly's Eye (HiRes
) Collaboration.  Primary research focus was the development---via extensive 
numerical modeling---of innovative data analysis techniques to compensate 
for severely asymmetric data resolution and distribution. Also served as the 
liaison between the collaboration and the Utah Center for High Performance 
Computing.

{\em Site Director} \hfill {\bf June 1999 - September 2001}\\
Oversaw the management of data acquisition for one (out of two) of the HiRes
observatories.  Duties included assuring that approximately 40 data-taking 
shifts were filled each month, management (transfers and archives) of 
$\sim1$~TB of data, development of automated data processing chains,
and the development of new fail-safe systems to protect 
photosensitive equipment from catastrophic system failures.

{\bf Harvard Medical School}, Belmont, Massachusetts, USA

\vspace{-.3cm}
{\em Research Consultant} \hfill {\bf April 2002 - January 2005}\\
Instigated and then carried out recruitment for a National Institutes of Health
funded research study in Salt Lake City with researchers from Harvard 
University and McLean Hospital.
 
\section{\sc Publications}

Abbasi, R.U.\ {\it et al.} {\bf First Upper Limits on the Radar Cross Section of Cosmic-Ray Induced Extensive Air Showers}. [arXiv:1603.05217 [astro-ph.IM]]

Aartsen, M.G.\ {\it et al.} {\bf Search for Correlations between the Arrival Directions of IceCube Neutrino Events and Ultrahigh-Energy Cosmic Rays Detected by the Pierre Auger Observatory and the Telescope Array}. JCAP {\bf 1601} 037 (2016). [arXiv:1511.09408 [astro-ph.HE]]

Abbasi, R.U.\ {\it et al.} {\bf Measurement of the Proton-Air Cross Section with Telescope Array’s Middle Drum Detector and Surface Array in Hybrid Mode}. Phys.\ Rev.\ {\bf D92} 032007 (2015). \newline
[arXiv:1505.01860 [astro-ph.HE]]

Abbasi, R.U.\ {\it et al.} {\bf The Hybrid Energy Spectrum of Telescope Array’s Middle Drum Detector and Surface Array}. Astropart.\ Phys.\ {\bf 68} 27 (2015). 

Shin, B.K.\ {\it et al.} {\bf Gain Monitoring of Telescope Array Photomultiplier Cameras for the First 4 Years of Operation}. Nucl.\ Instrum.\ Meth.\ {\bf A768} 96 (2014). 

Aab, A.\ {\it et al.} {\bf Searches for Large-Scale Anisotropy in the Arrival Directions of Cosmic Rays Detected above Energy of \boldmath$10^{19}$~eV at the Pierre Auger Observatory and the Telescope Array}. Astrophys.\ J.\ {\bf 794} 2 (2014). [arXiv:1409.3128 [astro-ph.HE]] 

Abbasi, R.U.\ {\it et al.} {\bf Study of Ultra-High Energy Cosmic Ray Composition Using Telescope Array's Middle Drum Detector and Surface Array in Hybrid Mode}. Astropart.\ Phys.\ {\bf 64} 49 (2014). \newline
[arXiv:1408.1726 [astro-ph.HE]]

Abbasi, R.U.\ {\it et al.} {\bf A Northern Sky Survey for Point-Like Sources of EeV Neutral Particles with the Telescope Array Experiment}. Astrophys.\ J.\ {\bf 804} 2, 133 (2014). [arXiv:1407.6145 [astro-ph.HE]]

Abbasi, R.U.\ {\it et al.} {\bf Indications of Intermediate-Scale Anisotropy of Cosmic Rays with Energy Greater Than 57 EeV in the Northern Sky Measured with the Surface Detector of the Telescope Array Experiment}. Astrophys.\ J.\ {\bf 790} L21 (2014). [arXiv:1404.5890 [astro-ph.HE]]

Abbasi, R.U.\ {\it et al.} {\bf CORSIKA Simulation of the Telescope Array Surface Detector}({\it corresponding author: B.T. Stokes}). [arXiv:1403.0644 [astro-ph.IM]]

Abu-Zayyad, T.\ {\it et al.} {\bf Correlations of the Arrival Directions of Ultra-high Energy Cosmic Rays with Extragalactic Objects as Observed by the Telescope Array Experiment}. Astrophys.\ J.\ {\bf 777} 88 (2013). [arXiv:1306.5808 [astro-ph.HE]]

Abu-Zayyad, T.\ {\it et al.} {\bf Energy Spectrum of Ultra-High Energy Cosmic Rays Observed with the Telescope Array Using a Hybrid Technique}. Astropart.\ Phys.\ {\bf 61} 93 (2015).  [arXiv:1305.7273 [astro-ph.HE]]

Abu-Zayyad, T.\ {\it et al.} {\bf The Energy Spectrum of Ultra-High-Energy Cosmic Rays Measured by the Telescope Array FADC Fluorescence Detectors in Monocular Mode}. Astropart.\ Phys.\ {\bf 48} 16 (2013). [arXiv:1305.6079 [astro-ph.HE]]

\newpage

Abu-Zayyad, T.\ {\it et al.} {\bf Upper limit on the flux of photons with energies above \boldmath$10^{19}$~eV using the Telescope Array surface detector}. Phys.\ Rev.\ {\bf D88} 112005 (2013). [arXiv:1304.5614 [astro-ph.HE]]  

Abu-Zayyad, T.\ {\it et al.} {\bf Search for Anisotropy of Ultra-High Energy Cosmic Rays with the Telescope Array Experiment}. Astrophys.\ J.\ {\bf 757} 26 (2012).  [arXiv:1205.5984 [astro-ph.IM]]

Abu-Zayyad, T.\ {\it et al.} {\bf The Cosmic Ray Energy Spectrum Observed with the Surface Detector of the Telescope Array Experiment} ({\it corresponding author: B.T. Stokes}). Astrophys.\ J.\ {\bf 768} L1 (2013). [arXiv:1205.5067 [astro-ph.IM]]

Abu-Zayyad, T.\ {\it et al.} {\bf The Energy Spectrum of Telescope Array's Middle Drum Detector and the Direct Comparison to the High Resolution Fly's Eye Experiment}. AstroPart.\ Phys.\ {\bf 39-40} 109 (2012). [arXiv:1202.5141 [astro-ph.IM]]

Abu-Zayyad, T.\ {\it et al.} {\bf The surface detector array of the Telescope Array experiment}. Nucl. Instrum. Meth. {\bf A689} 87 (2012). [arXiv:1201.4964 [astro-ph.IM]]

Tokuno, H.\ {\it et al.} {\bf New air fluorescence detectors employed in the Telescope Array experiment}. Nucl. Instrum. Meth. {\bf A676} 54 (2012). [arXiv:1201.0002 [astro-ph.IM]] 

Stokes, B.T.\ {\it et al.} {\bf Dethinning Extensive Air Shower Simulations in CORSIKA} ({\it corresponding author: B.T. Stokes}). Astropart.\ Phys.\ {\bf 35} 759 (2012). [arXiv:1104:3182 [astro-ph.IM]]

Stokes, B.T.\ {\it et al.} {\bf Simple Parallelization Scheme for Extensive Air Shower Simulations} ({\it corresponding author: B.T. Stokes}). [arXiv:1103.4643 [astro-ph.IM]]

Abbasi, R.U.\ {\it et al.} {\bf Analysis of large-scale anisotropy of ultra-high energy cosmic rays in HiRes data}. Astrophys.\ J.\ {\bf 713} L64 (2010). [arXiv:1002.1444 [astro-ph.HE]]

Abbasi, R.U.\ {\it et al.} {\bf Indications of Proton-Dominated Cosmic Ray Composition above 1.6 EeV}. Phys.\ Rev.\ Lett.\ {\bf 104} 161101 (2010). [arXiv:0910.4184 [astro-ph.HE]]

Abbasi, R.U.\ {\it et al.} {\bf Measurement of the Flux of Ultra High Energy Cosmic Rays by the Stereo Technique}.  Astropart.\ Phys.\ {\bf 32} 53 (2009). [arXiv:0904.4500 [astro-ph.HE]] 

Abbasi, R.U.\ {\it et al.} {\bf Search for Correlations between HiRes Stereo Events and Active Galactic Nuclei}. Astropart.\ Phys.\ {\bf 30} 175 (2008).
[arXiv:0804.0382 [astro-ph]]

Abbasi, R.U.\ {\it et al.} {\bf An Upper Limit on the Electron-Neutrino Flux
from the HiRes Detector}. Astrophys.\ J.\ {\bf 684} 790 (2008). 
[arXiv:0803.0554v1 [astro-ph]]

Chang, L.\ {\it et al.} {\bf Antiretroviral Treatment is Associated with 
Increased Attentional Load Dependent Activation in Patients with HIV.} NeuroImmune.\ Pharm.\ {\bf 3} 95 (2008). 

Gorham, P.W.\ {\it et al.} {\bf Observations of Microwave Continuum Emission 
from Air Shower Plasmas}. Phys.\ Rev.\ {\bf D78} 032007 (2008).
[arXiv:0705.2589v1 [astro-ph]]

Abbasi, R.U. {\it et al.} {\bf Observation of the GZK Cutoff by the HiRes
Experiment} Phys.\ Rev.\ Lett.\ {\bf 100} 101101 (2008).
[arXiv:astro-ph/0703099]

Abbasi, R.U. {\it et al.} {\bf Alternative Methods to Finding Patterns in 
HiRes Stereo Data} Astropart.\ Phys.\ {\bf 28} 385 (2007).
[arXiv:astro-ph/0702361]

Abbasi, R.U. {\it et al.} {\bf Studies of Systematic Uncertainties in the 
Estimation of the Monocular Aperture of the HiRes Experiment} Astropart.\ 
Phys.\ {\bf 27} 370 (2007). 
[arXiv:astro-ph/0607094]

Abbasi, R.U. {\it et al.} {\bf A Likelihood Method for Measuring the 
Ultrahigh Cosmic Ray Composition} Astropart.\ Phys.\ {\bf 26} 28 (2006). 
[arXiv:astro-ph/0604558]

Bergman, D.R. {\it et al.}  {\bf Can Experiments Studying Ultrahigh Energy
Cosmic Rays Measure the Evolution of the Sources?} [arXiv:astro-ph/0603797]

Barwick, S.W. {\it et al.}  {\bf Constraints on Cosmic Neutrino Fluxes from the
ANITA Experiment} Phys.\ Rev.\ Lett.\ {\bf 96} 171101 (2006).
[arXiv:astro-ph/0512265]

Abbasi, R.U. {\it et al.}  {\bf Search for Point-Like Sources of Cosmic Rays 
with Energies above \boldmath$10^{18.5}$~eV in the HiRes-I Monocular Data-Set} 
Astropart. Phys. {\bf 27} 512 (2007). 
[arXiv:astro-ph/0507663]

Abbasi, R.U. {\it et al.}  {\bf Search for Cross-Correlations of Ultra-High 
Energy Cosmic Rays with BL Lacertae Objects} Astrophys.\ J.\ {\bf 636} 680 
(2006). 
[arXiv:astro-ph/0507120]

Abbasi, R.U. {\it et al.}  {\bf A Measurement of Time-Averaged Aerosol Optical
Depth Using Air-Showers Observed in Stereo by HiRes}. Astropart.\ Phys.\
{\bf 25} 93 (2006). 
[arXiv:astro-ph/0601091]

Abbasi, R.U. {\it et al.}  {\bf Techniques for Measuring Atmospheric Aerosols 
for Air Fluorescence Experiments}. Astropart.\ Phys.\ {\bf 25} 74 (2006).
[arXiv:astro-ph/0512423]

Abbasi, R.U. {\it et al.}  {\bf Observation of the Ankle and Evidence for a
High-Energy Break in the Cosmic Ray Spectrum}. Phys.\ Lett.\ {\bf B619} 280 
(2005). [arXiv:astro-ph/0501317]

Abbasi, R.U. {\it et al.}  {\bf Search for Point Sources of Ultra-High Energy 
Cosmic Rays above \boldmath$4.0\times 10^{19}$~eV Using a Maximum Likelihood 
Ratio Test}. Astrophys.\ J.\ {\bf 623} 164 (2005).
[arXiv:astro-ph/0412617]

Abbasi, R.U. {\it et al.}  {\bf A Study of the Composition of Ultra High 
Energy Cosmic Rays Using the High Resolution Fly's Eye}.  Astrophys.\ J.\ 
{\bf 622} 910 (2005).
[arXiv:astro-ph/0407622]

Abbasi, R.U. {\it et al.}  {\bf A Search for Clustering in the HiRes-I 
Monocular Data above \boldmath$10^{19.5}$~eV} 
({\it corresponding author: B.T. Stokes}). 
Astropart.\ Phys.\ {\bf 22} 139 (2004). 
[arXiv:astro-ph/0404366]

Abbasi, R.U. {\it et al.}  {\bf Study of Small-Scale Anisotropy of Ultra-High 
Energy Cosmic Rays Observed in Stereo by HiRes}. Astrophys.\ J.\ {\bf 610} L73
(2004). 
[arXiv:astro-ph/0404137]

Abbasi, R.U. {\it et al.}  { \bf Search for Global Dipole Enhancements in the 
HiRes-1 Monocular Data above \boldmath$10^{18.5}$~eV }
({\it corresponding author: B.T. Stokes}).
Astropart.\ Phys.\ {\bf 21} 111 (2004).  
[arXiv:astro-ph/0309457]

Stokes, B.T., C.C.H. Jui, and J.N. Matthews.  {\bf Using Fractal 
Dimensionality in 
the Search for Source Models of Ultra-High Energy Cosmic Rays}
({\it corresponding author: B.T. Stokes})
Astropart.\ Phys.\ {\bf 21} 95 (2004).
[arXiv:astro-ph/0307491]

Abbasi, R.U. {\it et al.}  {\bf  Monocular Measurement of the Spectrum of 
UHE Cosmic Rays by the FADC Detector of the HiRes Experiment.}  
Astropart.\ Phys.\ {\bf 23} 157 (2005).
[arXiv:astro-ph/0208301]  

Abbasi, R.U. {\it et al.}  {\bf Measurement of the Flux of Ultra-high Energy 
Cosmic Rays from Monocular Observations by the High Resolution Fly's Eye 
Experiment.}  Phys.\ Rev.\ Lett.\ {\bf 92} 151101 (2004). \newline
[arXiv:astro-ph/0208243]

Sadowski, P.A. {\it et al.} {\bf Geometry and Optics Calibration of Air 
Fluorescence Detectors using Star Light.}  Astropart.\ Phys.\ {\bf 18} 237 
(2002).

\section{\sc Conference Presentations}

Stokes, B.T. {\it et al.} 2014 {\bf  Anisotropy in Cosmic Ray Arrival Directions Observed by the Telescope Array} American Physical Society April Meeting, Savannah, Georgia, USA. April 2014.

Stokes, B.T. {\it et al.} 2013. {\bf A Comparison between Hadronic Interaction Models and Observations by the Telescope Array}. 33$^{\rm rd}$ International Cosmic Ray Conference, Rio de Janeiro, Brazil. July 2013.

Stokes, B.T. {\it et al.} 2013 {\bf Measurement of the Ultra-High Energy Cosmic Ray Spectrum by the Telescope Array Surface Detector}. American Physical Society April Meeting, Denver, Colorado, USA. April 2013.

Stokes, B.T. {\it et al.} 2012. {\bf The Telescope Array: Current status and future prospects}.  {\it Invited Talk}. Annual Meeting of the Four Corners Section of the APS, Socorro, New Mexico, USA. October 2012.

Stokes, B.T. {\it et al.} 2012. {\bf Measuring the Ultra-High Energy Cosmic Ray Energy Spectrum with the Telescope Array}. American Physical Society April Meeting, Atlanta, Georgia, USA. April 2012. 

Stokes, B.T. {\it et al.} 2012. {\bf Using CORSIKA to quantify Telescope 
Array surface detector response}. International Symposium on Future Directions
in UHECR Physics, Geneva, Switzerland, February 2012.

Stokes, B.T. {\it et al.} 2011. {\bf Using CORSIKA to quantify Telescope 
Array surface detector response}. 32$^{\rm nd}$ International Cosmic Ray 
Conference, Beijing, China. August 2011.

Stokes, B.T. {\it et al.} 2011. {\bf Measurement of the Energy Spectrum by the 
Telescope Array Surface Detector}. 32$^{\rm nd}$ International Cosmic Ray 
Conference, Beijing, China. August 2011.

Stokes, B.T. {\it et al.} 2011. {\bf Expansion Plans for the Telescope Array Cosmic Ray Observatory}. American Physical Society April Meeting, Anaheim, California, USA. May 2011. 

Stokes, B.T. {\it et al.} 2010.  {\bf Measurement of UHECR energy spectrum by the Telescope Array Surface Detector}. International Symposium on the Recent Progress of Ultra-high Energy Cosmic Ray Observation, Nagoya, Japan, December 2010.

Stokes, B.T. {\it et al.} 2010.  {\bf The Telescope Array Experiment}. Annual Meeting of the Four Corners Section of the APS, Ogden, Utah, USA, October 2010.

Stokes, B.T. {\it et al.} 2010. {\bf Extensive air shower simulation for the Telescope Array surface detector}. XVI International Symposium on Very High Energy Cosmic Ray Interaction, Batavia, Illinois, USA.  June 2010.

Stokes, B.T. {\it et al.} 2010. {\bf The Telescope Array Experiment}. Snowbird
Workshop on Particle Astrophysics, Astronomy, \& Cosmology, Snowbird, Utah, USA.
March 2010.

Stokes, B.T. {\it et al.} 2010. {\bf Hybrid Observation with the Telescope Array Observatory}. American Physical Society April Meeting, Washington DC, USA.
February 2010.

Stokes, B.T. {\it et al.} 2009. {\bf Using CORSIKA to quantify Telescope 
Array surface detector response}. 31$^{\rm st}$ International Cosmic Ray 
Conference, (\L\'od\'z), Poland. July 2009.

Stokes, B.T.  2009.  {\bf Extensive air shower simulation: \boldmath$10^{19}$ eV
and beyond}. American Physical Society April Meeting, Denver, Colorado, USA.
May 2009.

Stokes, B.T. {\it et al.} 2007. {\bf Optimizing the Combination of 
Spiral-In/Out Images for BOLD and Perfusion fMRI} 13$^{\rm th}$ Annual Meeting 
of Human Brain Mapping.  Chicago, Illinois, USA.  June 2007

Stokes, B.T. {\it et al.}  2006. 
{\bf Characterization of Microwave Continuum Emission from UHECR Extensive Air 
Showers}. Proc.\ of EBHU (Kashiwa) {\bf 1} 39 (2006).

Stokes, B.T.  2004.  {\bf The Search for Anisotropy in the Arrival 
Directions of Ultra-High Energy Cosmic Rays Observed by the High Resolution 
Fly's Eye Detector in Monocular Mode}. 
Cosmic Ray International Seminar---GZK and Surroundings, Catania, Italy.
May 2004. Nucl.\ Phys.\ B (Proc.\ Suppl.) {\bf 136} 52 (2004).
[arXiv:astro-ph/0409377]

Stokes, B.T.  2004.  {\bf The Search for Anisotropy in the Arrival 
Directions of Ultra-High Energy Cosmic Rays Observed by the High Resolution 
Fly's Eye Detector in Monocular Mode}. 
American Physical Society April Meeting, Denver, Colorado, USA.
May 2004.

Stokes, B.T., C.C.H. Jui, and J.N. Matthews.  2004.  {\bf Using Fractal 
Dimensionality in the Search for Anisotropy of Ultra-High Energy Cosmic Rays}. 
Hawaii International Conference on the Sciences, Honolulu, Hawaii, USA.
January 2004.

Stokes, B.T., C.C.H. Jui, and J.N. Matthews.  2003.  {\bf Using Fractal 
Dimensionality in the Search for Anisotropy of Ultra-High Energy Cosmic Rays}. 
American Physical Society Four Corners Sectional Meeting, Tempe, Arizona, USA.
October 2003.

Stokes, B.T., C.C.H. Jui, and J.N. Matthews.  2003.  {\bf Using Fractal 
Dimensionality in the Search for Anisotropy of Ultra-High Energy Cosmic Rays}. 
Proc.\ of 28$^{\rm th}$ ICRC (Tsukuba) {\bf 1} 715 (2003).

Bellido, J., J. Belz, B. Dawson, M. Kirn, and  B.T. Stokes for the HiRes
Collaboration.  2003.  {\bf Anisotropy Studies of Ultra-High Energy Cosmic 
Rays Using Monocular Data Collected by the High-Resolution Fly's Eye (HiRes)}.
 Proc.\ of 28$^{\rm th}$ ICRC (Tsukuba) {\bf 1} 425 (2003).

Stokes, B.T. for the HiRes Collaboration.  2002.  {\bf Autocorrelation in the 
Highest Energy Event from the High Resolution Fly's Eye}.  American Physical
Society Four Corners Sectional Meeting, Salt Lake City, Utah, USA.  October
2002.   

Bellido, J., J. Belz, B. Dawson, M. Schindel, and B.T. Stokes for the HiRes
Collaboration.  2001.  {\bf Aniso-tropy Studies of Ultra-High Energy Cosmic 
Rays as Observed by HiRes}.  Proc.\ of 27$^{\rm th}$ ICRC (Hamburg) {\bf 1} 364
(2001).

Stokes, B.T. for the HiRes Collaboration.  1999.   {\bf Study of Anisotropy in 
Arrival Directions in the Highest Energy Cosmic Rays from High Resolution Fly's
Eye Results}.  26$^{\rm th}$ International Cosmic Ray Conference, Salt Lake 
City, Utah, USA. August 1999.

\section{\sc Colloquia and Seminars}

Brigham Young University, Department of Physics and Astronomy Colloquium,
October 2013. {\bf Cosmic Ray Research at the University of Utah}

Los Alamos National Laboratory, ISR Division Seminar, March 2006. 
{\bf Ultra-High Energy Cosmic Rays: Past, Present, and Future}

University of Hawai`i at M\={a}noa, 
Department of Physics and Astronomy Colloquium, 
Feburary 2005.  {\bf A Search for Anisotropy in the Arrival Directions 
of Ultra High Energy Cosmic Rays Observed by the High Resolution Fly's Eye 
Detector}

University of Utah, Department of Physics Special Seminar, January 2005.
{\bf A Search for Anisotropy in the Arrival Directions 
of Ultra High Energy Cosmic Rays Observed by the High Resolution Fly's Eye 
Detector}

University of Hawai`i at M\={a}noa, 
Department of Physics and Astronomy Colloquium, 
April 2004.  {\bf Using the High Resolution Fly's Eye to Help Solve the 
Mystery of Ultra-High Energy Cosmic Rays}

Los Alamos National Laboratory, P23 High Energy Physics Seminar, March 2004.
{\bf Using the High Resolution Fly's Eye to Help Solve the Mystery of
Ultra-High Energy Cosmic Rays}

University of Nijmegen, High Energy Physics Seminar, February 2004.
{\bf Using the High Resolution Fly's Eye to Help Solve the Mystery of
Ultra-High Energy Cosmic Rays}

Brigham Young University, Department of Physics and Astronomy Colloquium,
January 2004. {\bf Using the High Resolution Fly's Eye to Help Solve the 
Mystery of Ultra-High Energy Cosmic Rays}

University of Utah, High Energy Physics Seminar, November 2003.
{\bf Using the High Resolution Fly's Eye to Help Solve the Mystery of
Ultra-High Energy Cosmic Rays}

Columbia University, High Energy Physics Seminar, October 2003.
{\bf Using the High Resolution Fly's Eye to Help Solve the Mystery of
Ultra-High Energy Cosmic Rays}

\end{resume}
\end{document}


