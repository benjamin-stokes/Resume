%-------------------------------------------------------------------------------
%PACKAGES AND OTHER DOCUMENT CONFIGURATIONS
%-------------------------------------------------------------------------------

\documentclass[10pt,letterpaper]{article} % Font size (10-12pt) and
                                % paper size (a4paper, letterpaper,
                                % legalpaper, etc)

% Copyright (c) 2012 Cies Breijs
%
% The MIT License
%
% Permission is hereby granted, free of charge, to any person obtaining a copy
% of this software and associated documentation files (the "Software"), to deal
% in the Software without restriction, including without limitation the rights
% to use, copy, modify, merge, publish, distribute, sublicense, and/or sell
% copies of the Software, and to permit persons to whom the Software is
% furnished to do so, subject to the following conditions:
%
% The above copyright notice and this permission notice shall be included in
% all copies or substantial portions of the Software.
%
% THE SOFTWARE IS PROVIDED "AS IS", WITHOUT WARRANTY OF ANY KIND, EXPRESS OR
% IMPLIED, INCLUDING BUT NOT LIMITED TO THE WARRANTIES OF MERCHANTABILITY,
% FITNESS FOR A PARTICULAR PURPOSE AND NONINFRINGEMENT. IN NO EVENT SHALL THE
% AUTHORS OR COPYRIGHT HOLDERS BE LIABLE FOR ANY CLAIM, DAMAGES OR OTHER
% LIABILITY, WHETHER IN AN ACTION OF CONTRACT, TORT OR OTHERWISE, ARISING FROM,
% OUT OF OR IN CONNECTION WITH THE SOFTWARE OR THE USE OR OTHER DEALINGS IN THE
% SOFTWARE.

%%% LOAD AND SETUP PACKAGES

\usepackage[margin=0.75in]{geometry} % Adjusts the margins

\usepackage{multicol} % Required for multiple columns of text

\usepackage{mdwlist} % Required to fine tune lists with a inline headings and indented content

\usepackage{relsize} % Required for the \textscale command for custom small caps text

\usepackage{hyperref} % Required for customizing links
\usepackage{xcolor} % Required for specifying custom colors
\definecolor{dark-blue}{rgb}{0.15,0.15,0.45} % Defines the dark blue color used for links
\hypersetup{colorlinks,linkcolor={dark-blue},citecolor={dark-blue},urlcolor={dark-blue}} % Assigns the dark blue color to all links in the template

\usepackage{tgtermes} % Use the TeX Gyre Pagella font throughout the document
\usepackage[T1]{fontenc}
\usepackage{microtype} % Slightly tweaks character and word spacings for better typography

\pagestyle{empty} % Stop page numbering

%----------------------------------------------------------------------------------------
%	DEFINE STRUCTURAL COMMANDS
%----------------------------------------------------------------------------------------

\newenvironment{indentsection} % Defines the indentsection environment which indents text in sections titles
{\begin{list}{}{\setlength{\leftmargin}{\newparindent}\setlength{\parsep}{0pt}\setlength{\parskip}{0pt}\setlength{\itemsep}{0pt}\setlength{\topsep}{0pt}}}{\end{list}}

\newcommand*\maintitle[2]{\noindent{\LARGE \textbf{#1}}\ \ \ \emph{#2}\vspace{0.3em}} % Main title (name) with date of birth or subtitle

\newcommand*\roottitle[1]{\subsection*{#1}\vspace{-0.3em}\nopagebreak[4]} % Top level sections in the template

\newcommand{\headedsection}[3]{\nopagebreak[4]\begin{indentsection}\item[]\textscale{1.1}{#1}\hfill#2#3\end{indentsection}\nopagebreak[4]} % Section title used for a new employer

\newcommand{\headedsubsection}[3]{\nopagebreak[4]\begin{indentsection}\item[]\textbf{#1}\hfill\emph{#2}#3\end{indentsection}\nopagebreak[4]} % Section title used for a new position

\newcommand{\bodytext}[1]{\nopagebreak[4]\begin{indentsection}\item[]#1\end{indentsection}\pagebreak[2]} % Body text (indented)

\newcommand{\inlineheadsection}[2]{\begin{basedescript}{\setlength{\leftmargin}{\doubleparindent}}\item[\hspace{\newparindent}\textbf{#1}]#2\end{basedescript}\vspace{-1.7em}} % Section title where body text starts immediately after the title

\newcommand*\acr[1]{\textscale{.85}{#1}} % Custom acronyms command

\newcommand*\bull{\ \ \raisebox{-0.365em}[-1em][-1em]{\textscale{4}{$\cdot$}} \ } % Custom bullet point for separating content

\newlength{\newparindent} % It seems not to work when simply using \parindent...
\addtolength{\newparindent}{\parindent}

\newlength{\doubleparindent} % A double \parindent...
\addtolength{\doubleparindent}{\parindent}

\newcommand{\breakvspace}[1]{\pagebreak[2]\vspace{#1}\pagebreak[2]} % A custom vspace command with custom before and after spacing lengths
\newcommand{\nobreakvspace}[1]{\nopagebreak[4]\vspace{#1}\nopagebreak[4]} % A custom vspace command with custom before and after spacing lengths that do not break the page

\newcommand{\spacedhrule}[2]{\breakvspace{#1}\hrule\nobreakvspace{#2}} % Defines a horizontal line with some vertical space before and after it
 % Include structure.tex which contains packages
                    % and document layout definitions

\hyphenation{Some-long-word} % Specify custom hyphenation points in
                             % words with dashes where you would like
                             % hyphenation to occur, or alternatively,
                             % don't put any dashes in a word to stop
                             % hyphenation altogether

\begin{document}

%-------------------------------------------------------------------------------
%NAME AND CONTACT INFORMATION
%-------------------------------------------------------------------------------

\maintitle{Benjamin Stokes, Ph.D.}{}

\noindent\textsmaller{+}1 (801) 839-8993\bull
\href{mailto:stokes@cosmic.utah.edu}{stokes@cosmic.utah.edu}\bull
\href{https://github.com/benjamin-stokes}
{github.com/benjamin-stokes}\\
\href{http://www.linkedin.com/in/benjamintstokes}
{www.linkedin.com/in/benjamintstokes}\bull
\href{https://plus.google.com/+BenjaminStokesPhD}
{plus.google.com/\textsmaller{+}BenjaminStokesPhD}

\spacedhrule{0.9em}{-0.4em}

%-------------------------------------------------------------------------------
%SUMMARY SECTION
%-------------------------------------------------------------------------------

\roottitle{Summary}

\vspace{-0.3em}
\bodytext{\textit{I am a problem solver with extensive experience
    collecting, processing, and analyzing large amounts of data in
    distributed settings.  Through a high degree of creative
    persistence, I have successfully tackled extremely difficult
    problems.  Specific areas of expertise include:}
  \begin{multicols}{2}  % Start a two-column layout
  \begin{itemize}
\item Big data management
\vspace{-0.3em}\item Data reduction and analysis
\vspace{-0.3em}\item Statistical methods and machine learning
\vspace{-0.3em}\item Algorithm development and implementation
\vspace{-0.3em}\item Distributed systems
\vspace{-0.3em}\item Advanced troubleshooting
\item Software architecture and design
\vspace{-0.3em}\item Monte Carlo modeling and simulations
\vspace{-0.3em}\item \acr{HPC} design and administration
\vspace{-0.3em}\item Strategic problem solving
\vspace{-0.3em}\item International collaboration
\vspace{-0.3em}\item Public presentations
\vspace{-0.3em}  \end{itemize}
\end{multicols}}

\spacedhrule{0.5em}{-0.4em}

%-------------------------------------------------------------------------------
%SKILLS SECTION
%-------------------------------------------------------------------------------

\roottitle{Skills}

\inlineheadsection
    {Computer Languages:}
    {\acr{C}, Bash, \acr{R}, Go, Python, \LaTeX, \acr{C++},
      \acr{FORTRAN}, My\acr{SQL}, Tcsh, Visual Basic, Wiki, \acr{XML}}

    \inlineheadsection
        {Data Management, Statistical Analysis, and Machine Learning:}
        {C, Bash, No\acr{SQL}, \acr{R}, Python, My\acr{SQL}, Excel}

    \inlineheadsection
        {Distributed Systems and Virtualization:}
        {\acr{PBS}, Rocks, VirtualBox, \acr{VM}ware, Hadoop,
          MapReduce, Spark, \acr{AWS}}

    \inlineheadsection
        {Frameworks, Environments, and Operating Systems:}
        {Linux, Emacs, Windows, \acr{SVN}, Git, Unix, Xilinx}

    \inlineheadsection
        {Natural languages:}
        {English \textit{(native proficiency)}, Russian
          \textit{(elementary proficiency)}, French \textit{(beginner)}}

\spacedhrule{2.0em}{-0.4em}

%-------------------------------------------------------------------------------
%EXPERIENCE SECTION
%-------------------------------------------------------------------------------

\roottitle{Experience}

\headedsection
{\href{http://www.utah.edu}{University of Utah}}
{\textsc{Salt Lake City, Utah}} {

\headedsubsection
{Postdoctoral Research Associate}
{May '10 -- Mar '16}
{\bodytext{I was a research collaborator with the
    \href{http://www.telescopearray.org}{Telescope Array} (\acr{TA})
    cosmic ray observatory. As a one-of-a-kind facility, the 150
    scientists and technologists in the \acr{TA} collaboration are
    responsible for design and maintenance of remote facilities (in
    Millard County, Utah), data acquisition, data storage, data
    analysis and public dissemination of resulting scientific
    knowledge.  This mission has required the \acr{TA} collaboration
    to develop extensive computational resources and has resulted in
    groundbreaking discoveries about the origin and composition of
    cosmic rays. My principal achievements included:
    \begin{itemize}
    \item \textit{\textbf {Developing a
          \href{http://arxiv.org/abs/1103.4643}{technique} for mapping
          and reducing parallel computations.}} I pursued this effort
      with the aim of open-ended scalability and robust fault
      protection. The resulting software could be described as a
      highly specialized reinvention of MapReduce implemented on Linux
      clusters with Bash scripting employing a high degree of
      concurrency.
    \item \textit{\textbf{Designing the primary framework for the
          Monte Carlo simulation of the \acr{TA} observatory.}}  This
      effort entailed integrating 40 years worth of legacy code
      (\acr{FORTRAN} and \acr{C}), writing 20,000 lines of new code
      (\acr{C}, \acr{C++} and No\acr{SQL}), and innovating an entirely
      \href{http://arxiv.org/abs/1104.3182}{novel algorithmic
        approach}.  The
      \href{http://arxiv.org/abs/1403.0644}{resulting simulations} were
      \href{http://arxiv.org/abs/1205.5067}{unprecedented} in detail
      and accuracy.
   \item \textit{\textbf{Engaging in big data management.}}  In
     addition to the two Linux clusters described below, I built and
     manage a 50~TB Linux data server and manage a 10~TB remote data
     space. The simulation framework described above is capable of
     producing 500~GB per day.  In order to effectively utilize this
     data, I developed a keen sense of foresight towards data
     logistics.
    \item \textit{\textbf{Becoming a master troubleshooter.}}  The
      \acr{TA} observatory is operational 24 hours per day, 365 days
      per year.  With operational costs and capital outlays exceeding
      \$100,000 per month, it is imperative that any problems be
      addressed promptly. As a mid-level collaboration member, I
      developed the skills to rapidly troubleshoot a broad array of
      hardware and software problems.
    \item \textit{\textbf{Engaging in international collaboration.}}
      The \acr{TA} collaboration is 70\% Japanese and, in addition to
      the University of Utah, also has member institutions in South
      Korea, Belgium, and Russia.  Working in this diverse setting
      taught me to value, above all else, clear communication while
      respecting the cultural differences and sensitivities of those
      around me.
\end{itemize}}}}

\newpage

%-------------------------------------------------------------------------------
%NAME AND CONTACT INFORMATION
%-------------------------------------------------------------------------------

\maintitle{Benjamin Stokes, Ph.D.}{}

\noindent\textsmaller{+}1 (801) 839-8993\bull
\href{mailto:stokes@cosmic.utah.edu}{stokes@cosmic.utah.edu}\bull
\href{https://github.com/benjamin-stokes}
{github.com/benjamin-stokes}\\
\href{http://www.linkedin.com/in/benjamintstokes}
{www.linkedin.com/in/benjamintstokes}\bull
\href{https://plus.google.com/+BenjaminStokesPhD}
{plus.google.com/\textsmaller{+}BenjaminStokesPhD}

\spacedhrule{0.9em}{-0.2em}

%-------------------------------------------------------------------------------
%EXPERIENCE SECTION (cont.)
%-------------------------------------------------------------------------------

\roottitle{Experience (cont.)}

\headedsection
{\href{http://www.rutgers.edu}{Rutgers, The State University of New Jersey}}
{\textsc{Piscataway, New Jersey}} {

\headedsubsection
{Postdoctoral Associate}
{Jul '08 -- Apr '10}
{\bodytext{I first joined the \acr{TA} collaboration as a researcher
    with the affiliated group at Rutgers University.  While my
    position was officially in New Jersey, I spent most of my time on
    remote deployment in Utah. In addition to commencing the projects
    described in the previous section, relevant experience included:
    \begin{itemize}
    \item \textit{\textbf{Building and administering two 20-node
          \href{http://www.rocksclusters.org}{Rocks} Linux clusters.}}
      This proved to be an important test bed for later distributed
      computing  initiatives.
    \item \textit{\textbf{Learning to work remotely.}}  Spending the
      majority of my time 3000~km from the rest of my research group
      was an excellent opportunity to develop independence, clear
      communication, and self-motivation.
    \end{itemize}}}}

\headedsection
    {\href{http://manoa.hawaii.edu}{University of Hawai`i at M\={a}noa}}
    {\textsc{Honolulu, Hawai`i}}{

\headedsubsection
{Research Fellow}
{Jan '05 -- Aug '06}
{\bodytext{I spent 18 months in a research fellowship studying radio
    detection of cosmic rays and neutrinos as a member of both the
    \href{http://arxiv.org/abs/astro-ph/0512265}{ANITA} and
    \href{http://arxiv.org/abs/0705.2589}{AMBER} collaborations.
    Major activities included:
    \begin{itemize}
    \item \textit{\textbf{Designing and debugging embedded systems.}}
      I developed both embedded software (Linux and \acr{C}) and
      firmware (Xilinx).
    \item \textit{\textbf{Designing and running Monte Carlo
          simulations.}} I designed simulations both by writing new
      code (\acr{C}) and by adapting legacy code (\acr{C} and
      \acr{FORTRAN}) with the objective of studying different
      observation scenarios.
    \end{itemize}}}

\headedsubsection
{Junior Researcher}
{Sep '06 -- Jun '08}
{\bodytext{After my astrophysics research fellowship ended, I decided
    to remain in Honolulu and secured a position with the
    \href{http://hawaii.edu/mri/home_v6.htm}{Neuroscience and
      \acr{MRI} Research Program}.  My primary responsibilities
    included analyzing \acr{fMRI} data, administering a 25-node Linux
    cluster, and performing clinical duties.  Major milestones included:
    \begin{itemize}
    \item \textit{\textbf{Successfully executing a major domain
          change.}} In making the transition to neuroscience, I
      quickly learned a whole new nomenclature and worldview. In doing
      so, I developed extensive strategies for efficiently tackling
      unfamiliar problems.
    \item \textit{\textbf{Contributing to enterprise software
          development.}} After attending a two-week training course in
      North Carolina, I participated in development of the operational
      source code (\acr{C++}) of Siemens \acr{MRI} scanners.
    \end{itemize}}}}

\spacedhrule{0.8em}{-0.4em}

%-------------------------------------------------------------------------------
%EDUCATION SECTION
%-------------------------------------------------------------------------------

\roottitle{Education}

\headedsection
{University of Utah}
{\textsc{Salt Lake City, Utah}} {

\headedsubsection
{Doctor of Philosophy in Physics}
{}
{\bodytext{Dissertation:
    \href{http://www.cosmic-ray.org/thesis/stokes.html}{\textit{A
        Search for Anisotropy in the Arrival Directions of Ultra High
        Energy Cosmic Rays Observed by the High Resolution Fly's Eye
        Detector}}}}

\headedsubsection
{Bachelor of Arts in Physics}
{}{}}


\headedsection
{Timpview High School}
{\textsc{Provo, Utah}} {

\headedsubsection
{High School Diploma}
{}
{\bodytext{Valedictorian}}
}

\spacedhrule{1.0em}{-0.4em}

%-------------------------------------------------------------------------------
%AWARDS SECTION
%-------------------------------------------------------------------------------

\roottitle{Awards}

\bodytext{\begin{itemize}
\item Outstanding Postdoctoral Researcher, \textit{University of Utah
    Department of Physics and Astronomy}
\vspace{-0.3em}\item U.S.\ Presidential Scholar, \textit{White House}
\vspace{-0.3em}\item National Science Scholar, \textit{U.S.\
  Department of Education}
\vspace{-0.3em}\item Eagle Scout, \textit{Boy Scouts of America}
\end{itemize}}

\spacedhrule{1.0em}{-0.4em}

%-------------------------------------------------------------------------------
%OTHER INTERESTS SECTION
%-------------------------------------------------------------------------------

\roottitle{Other Interests}

\bodytext{Among many passions, I am a four-season mountaineer, a
  classically-trained double bassist, and an internationally published
  amateur photographer.}

\end{document}
